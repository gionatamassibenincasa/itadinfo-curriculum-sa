% -*- mode: latex; coding: utf8; TeX-master: curricolo_sa.tex -*-
% !TeX root = curricolo_sa.tex
% !TeX encoding = UTF-8 Unicode
% !Tex TS-program = LuaLaTeX
\documentclass[a4paper]{easychair}
\usepackage[italian]{babel}
\usepackage[autostyle,italian=guillemets,]{csquotes}
\usepackage[autostyle,italian=guillemets]{csquotes}
\usepackage[backend=biber,style=numeric,hyperref]{biblatex}
\addbibresource{curricolo_sa.bib}
%\usepackage{apacite}
\ifPDFTeX
  \usepackage[utf8]{inputenc}
\fi

%% Front Matter
%%
% Regular title as in the article class.
%
\title{Un curricolo per separare l'informatica dall'applimatica\thanks{Neologismo %
trovato in \href{https://aladdin.unimi.it/materiali/talk/2012_mirabilandia.pdf}%
{alcuni lucidi del prof. Mattia Monga} per indicare l'uso del computer o di applicativi specifici}\\ in un %
Liceo scientifico opzione scienze applicate}

\author{
Gionata Massi\institute{
  IIS Savoia Benincasa, Ancona (AN)
 }
}

\institute{
  IIS Savoia Benincasa, Ancona (AN)\\
  \email{gionata.massi@savoiabenincasa.it}
}

\authorrunning{Massi}

\titlerunning{Un curricolo per le Scienze Applicate}

\begin{document}

\maketitle

\begin{abstract}
Questo articolo presenta un curricolo per l'insegnamento dell'informatica
nel Liceo scientifico opzione scienze applicate che ambisce di introdurre
la materia come disciplina scientifico/ingegneristica\footnote{Harold Abelson nella
\href{https://youtu.be/-J_xL4IGhJA}{prima lezione del corso MIT~6.001
Structure and Interpretation of Computer Programs}
sembra essere in disaccordo sulla natura scientifica dell'informatica
% https://www.driverlesscrocodile.com/technology/the-wizard-1-harold-abelson-on-the-essence-of-computer-science-as-formalising-procedural-knowledge/
} piuttosto che una formazione professionale sull'uso dei software applicativi,
come talvolta accade nelle scuole italiane.

Il curricolo è orientato all'apprendimento dei fondamenti epistemologici
dell'informatica in accordo con i traguardi di competenza fissati 
nelle Indicazioni nazionali ma presenta una selezione di argomenti e contenuti
che sconcertano studenti, genitori e colleghi perché non rispondono alle attese di un
percorso di addestramento all'uso delle tecnologie informatiche.

La didattica di alcuni argomenti richie l'uso del PC e
del software\footnote{Nella scelta del software ci si è limitati
alle sole applicazioni rilasciate come software libero o, laddove non fosse
possibile, a sorgente aperto.} e questo è implicitamente richiesto dalla Indicazioni nazionali
e non è in contraddizione con gli obiettivi epistemologici.

Il curricolo è presentato come un estratto della progettazione didattica
nel primo biennio, nel secondo biennio e nel quinto anno
elencando un sottoinsieme di argomenti,
la collocazione all'interno dell'area tematica\footnote{Si vedano le Indicazioni nazionali~\cite{IlMinistro2010}},
% la motivazione principale per la scelta degli argomento,
i traguardi di competenza espressi come abilità verificabili e% i materiali didattici usati.
d eventuali annotazioni.

La sperimentazione del curricolo è stata effettuata tra gli anni scolastici
2018/19 e 2022/23 coinvolgendo due classi nel corso degli anni e, soprattutto
a causa della sopraggiunta emergenza sanitaria, non è stato possibile condurre analisi
comparative sul raggiungimento degli obiettivi di competenza rispetto alle
classi parallele.

In conclusione il curricolo qui proposto può essere utilizzato come un
catalogo di traguardi di apprendimento dal quale gli insegnanti possono attingere liberamente
per la progettazione e la programmazione didattica dell'informatica % che sia formativa e orientante
del proprio istituto; esso è un contributo verso l'insegnamento dell'informatica come disciplina
scientifica e tecnica piuttosto che come un'addestramento all'uso di strumenti tecnologici e
e può essere meritevole di ulteriore sperimentazione ma il passaggio dalla didattica dall'applimatica
a quello dell'informatica ha un costo troppo elevato per i singoli insegnanti
in assenza di materiali di studio in lingua italiana di elevata qualità,
auspicabilmente nella forma di Open Educational Resource (OER).
% Gli editori dei libri di testo scolastici e della comunità degli insegnanti
% d'informatica della scuola secondaria dovrebbero lavorare in tal senso con il supporto
% di coordinamento e revisione degli accademici.
\end{abstract}

% solo per ragioni tipografiche
\clearpage

\setcounter{tocdepth}{2}

{\small
\tableofcontents}

\section{Introduzione}
\label{sect:introduction}

Gli insegnanti di informatica italiani dei licei italiani, a differenza dei colleghi di molte
altre nazioni, non hanno sillabi, programmi nazionali e sistemi di valutazione
sviluppati per l'insegnamento della loro materia ma devono
progettare e programmare l'attività didattica mediando fra vincoli normativi nazionali,
quadri europei sulla competenza digitale\footnote{Si veda il DigComp 2.2 con oltre 250 competenze digitali. L'aggettivo \textit{digitale} probabilmente si riferisce alle dita che esercitano pressione sulla tastiera piuttosto che essere la traduzione del falso amico che in passato si traduceva con \textit{numerico}.} 
e altre  disparate esigenze.

La programmazione didattica disciplinare si realizza come se si
dovesse ottimizzare un modello di programmazione matematica mai formalizzato.
Le variabili decisionali sono gli argomenti e i contenuti
utili a sviluppare le competenze proprie dell'informatica.
I vincoli sono dati dalle Indicazioni nazionali~\cite{IlMinistro2010} e
dal Piano Triennale dell'Offerta Formativa della scuola.
La funzione obiettivo pondera le necessità del territorio,
l'adeguamento alle aspettative di studenti, genitori e altri insegnanti,
la presenza dei contenuti del libro di testo in adozione, laddove questo ci sia,
il riuso del materiale didattico già realizzato,
l'incentivo al conseguimento di certificati, rilasciati a pagamento, sull'uso di software applicativi,
specie nel caso in cui la scuola sia sede d'esame,
la conformità ai \textit{framework di competenze},
e le conoscenze degli insegnati stessi\footnote{Il requisito di accesso all'insegnamento per la classe di concorso \textit{A-41~--~Scienze e tecnologie informatiche} è molto più lasco rispetto a quelle di altre classi di concorso e molti insegnanti, specie quelli meno giovani, non hanno sostenuto alcun esame di informatica nel percorso universitario.}.

Nelle esperienze dell'autore, l'aspettativa dell'utenza e del corpo docente si concentra
verso un percorso di applimatica$\phantom{}^\star$ volto a fornire agli studenti
le sole abilità d'uso di qualche programma applicativo con il quale interagire tramite
un'interfaccia grafica e un comportamento \textit{What You See Is What You Get}\footnote{Questo termine indica che quando usi un'applicazione per modificare un documento, \textit{quello che vedi} nell'interfaccia di modifica \textit{è quello che ottieni} nell'interfaccia di visualizzazione. L'applicazione di modifica e quella di visualizzazione possono coincidere.} (WYSIWYG).
Nel dettaglio, la classe dei programmi applicativi richiesti dagli insegnanti,
include sempre un certo programma per l'elaborazione del testo e un altro per la predisposizione dei lucidi di supporto alla presentazione.
In aggiunta possono essere aggiunti i fogli elettronici e alcuni sistemi per la rappresentazione geometrica e/o per il calcolo numerico o simbolico.

I libri di testo del bienno sono spesso adatti a soddisfare questo genere di aspettative, anche per rispettare le Indicazioni nazionali,
e negli ultimi tre anni presentano un linguaggio di programmazione\footnote{%
Nel primo biennio i linguaggi di programmazione sono sostituiti da ambienti grafici per la programmazione visuale a blocchi e il termine \textit{programmazione} è sostituito da \textit{coding}},
spesso uno solo e frequentemente il C++\footnote{Bjarne Stroustrup sembra sorpreso dal fatto che il C++ sia ampiamente utilizzato nella didattica in quanto tale linguaggio non è il più compatto e pulito e nella genesi del linguaggio non vi sono elementi sulla natura didattica.% p. 17 C++ libreria standard
}.

Nelle nazioni anglofone, le uniche per le quali l'autore è riuscito ad effettuare
una ricerca sullo stato dell'arte, esistono vari curricula molto ben dettagliati
che fanno apparire il nostro processo di programmazione didattica disciplinare
empirica, eterogenea, orientata all'uso delle applicazioni e non supportata da
materiale di studio valido\footnote{%
Il sistema scolastico italiano ha molte peculiarità che lo rendono estremante formativo,
o almeno questa è l'impressione che se ne ricava quando si assiste ai colloqui pluridisciplinari
degli studenti che hanno frequentato un anno scolastico all'estero,
ma questa affermazione sembra non essere vera per l'informatica.}.

% Per la disamina dei percorsi didattici è rilevante notare che
% (a) spesso si tende a sostituire il termine \textit{informatica}
% con \textit{pensiero computazionale}\footnote{Sulla necessità di tale concetto si veda l'articolo \cite{Lodi2017}}\cite{Wing2006},
% (b) vi è una forte spinta verso tematiche, linguaggi e tecnologie in voga nel periodo in cui si scrive
% il curriculum\footnote{Si vedano ad esempio le recenti monografie sul tema, ad esempio~\cite{Kong2022}} e
% (c) i cicli di vita rispecchiano quelli delle tecnologie.


Ad esempio, negli Stati Uniti ci sono curricula e attività didattiche progettate in
dettaglio, come~\cite{Fisler2021} che propone scopi affini
alle nostre \textit{indicazioni nazionali}~\cite{IlMinistro2010}; le esperienze didattiche e i curricula sono
spesso raccolte in cataloghi quali~\cite{cs4all} e piattaforme di apprendimento online come Code.org~\cite{codeorg}
e Khan Academy~\cite{khan-academy} offrono corsi gratuiti basati sui curricula più noti
come lo Advanced Placement Computer Science~\cite{ap-csp}.
La comunità scientifica si occupa da tempo della didattica dell'informatica anche tramite lo sviluppo
di linguaggi di programmazione\footnote{Albeson e Stroustrup, per citare due creatori di linguaggi di programmazione, considerano il linguaggio di programmazione come uno strumento per esprimere le idee sotto forma di codice.} e ambienti per l'insegnamento\footnote{%
Seymour Papert e Mitchel Resnick sono così famosi tra gli insegnati italiani che è inutile citarli ma il loro contributo ora viene usato per spingere verso il \textit{coding} che pare cosa diversa dalla programmazione.} %
(\cite{Abelson1996} e \cite{Abelson2022}, \cite{Felleisen2018}). 
Anche in Gran Bretagna c'è molta attenzione all'insegnamento dell'informatica
e alla formazione professionale dei docenti\footnote{Un progetto di riferimento è~\cite{nc4ce} ma tante comunità e fondazioni
di produttori di hardware Open Source che propongono curricula completi.
}~\cite{Fowler2021}.
Di rilievo appaiono anche i progetti didattici della Nuova Zelanda e tra quelli dedicati all'informatica
vi è la \textit{Computer Science Field Guide}~\cite{UniComputerScienceEducationResearchGroupCanterbury2023},
una guida per insegnanti e studenti che è stata tradotta in varie lingue\footnote{La guida è tradotta in tedesco, spagnolo e polacco. Alla lista manca l'italiano e non sono segnalati progetti di traduzione nella lingua italiana.}.

% Si propone una di queste programmazioni costruita rilassando
% i vincoli sulle aspettative e sui testi, prevedendo attività
% alla lavagna e, se al PC, usando esclusivamente
% Software Libero e formati aperti.

% ambienti didattici interattivi - Interactive learning environments

Questo documento presenta nel par.~\ref{sec:curricolo} la proposta di curricolo
che l'autore ha progettato riducendo il peso delle scelte relative all'applimatica,
nel successivo par.~\ref{sec:sperimentazione} si illustra la sperimentazione e si
chiariscono le cause della mancata osservazione comparativa con altre classe del medesimo ciclo e
nel par.~\ref{sec:conclusioni} si determinano le condizioni dell'applicabilità del curricolo in base
alla sperimentazione.

%si tirano le somme..
%Nel par.~\ref{sec:b1} si propongono alcune abilit\`a
%pensate per il primo biennio, nel par.~\ref{sec:b2} quelle del secondo
%biennio e quinto anno, mentre nel par.~\ref{sec:conclusioni}
%si tirano le somme.

%%% INIZIO CURRICOLO
\section[Curricolo]{Curricolo}
\label{sec:curricolo}

Per esigenze di brevità, il curricolo è riportato in modo parziale e si rimanda il lettore alla pagina \url{https://github.com/gionatamassibenincasa/progettazione-didattica-informatica}.


\subsection{Primo biennio}
\label{sec:primo-biennio}

\subsubsection[Scrittura tecnico-scientifica]{Scrittura tecnico-scientifica al calcolatore}
\label{sec:scrittura-tecnico-scientifica}

\noindent\textbf{Area tematica}: Elaborazione digitale dei documenti (DE)

%\noindent\textbf{Abilità}:
    
\begin{enumerate}
  \item Identificare gli attori di una comunicazione
  \item Distinguere tra dati, informazioni e conoscenza
  \item Aggiungere l'autore e il titolo nel frontespizio di una relazione
  \item Suddividere un testo nei suoi elementi strutturali ed aggiungere i titoletti
  \item Scrivere tabelle, immagini e altri contenuti flottanti con didascalie
  \item Scrivere con elenchi ordinati, non ordinati e descrittivi
  \item Enfatizzare il testo selezionando un carattere tipografico opportuno
\end{enumerate}

L'argomento è introdotto per soddisfare sia il vincolo ministeriale sulla produzione di documenti elettronici.
La videoscrittura si colloca di buon diritto nell'applimatica per cui si è scelto l'argomento della comunicazione tecnico/scientifica
e un linguaggio di dominio specifico dotato di grammatica libera dal contesto.
Lo strumento di elaborazione del testo diventa così un linguaggio avente una grammatica espressa tramite il suo diagramma sintattico
e l'attività di videoscrittura diventa un'approccio ai linguaggi formali e alla programmazione.

\subsubsection[Dati e codifiche]{Dati e codifiche}
\label{sec:dati-e codifiche}

\noindent\textbf{Area tematica}: Architettura dei computer (AC)

%\noindent\textbf{Abilità}:
    
\begin{enumerate}
  \item Convertire un numero da base due a base dieci
  \item Convertire un numero da base dieci a base due
  \item Codificare un testo usando codifiche binarie dei caratteri
  \item Decodificare un testo codificato in binario
  \item Comprendere le esigenze che hanno condotto allo sviluppo dello standard UNICODE
  \item Codificare un'immagine monocromatica mediante run-length encoding
  \item Decodificare un'immagine monocromatica codificata mediante run-length encoding
\end{enumerate}

L'argomento è proposto per soddisfare la richiesta di introdurre lo studente alla codifica binaria
senza limitarsi ai codici codici ASCII e Unicode citati nelle Indicazioni nazionali.
Verrà ampiamente ripreso al quinto anno quando, nell'analisi degli algoritmi numerici,
sono trattate le codifiche in virgola mobile e gli errori di troncamento.

%La didattica si avvale delle attività proposte nel testo \textit{Computer Science Unplugged}~\cite{csu}.

%Lo studente non è introdotto alla sola rappresentazione del testo ma anche a quella dei numeri naturali e delle immagini raster. Al concetto di cifra binaria come elemento di codifica viene affiancato in modo intuitivo il concetto di \textit{bit} come unità di misura dell'informazione e la capacità di mettere in relazione la quantità di bit usati per rappresentare la stessa informazione usando codici diversi.

\subsubsection[Problemi e algoritmi]{Problemi, modelli, soluzioni e algoritmi}
\label{sec:problemi-e algoritmi}

\noindent\textbf{Area tematica}: Algoritmi e linguaggi di programmazione (AL)

%\noindent\textbf{Abilità}:
    
\begin{enumerate}
  \item Saper formalizzare un problema di ricerca
  \item Simulare l'esecuzione dell'algoritmo di ricerca lineare
  \item Simulare l'esecuzione dell'algoritmo di ricerca binaria
  \item Saper formalizzare il concetto di ordinamento di una sequenza
  \item Simulare l'algoritmo di ordinamento per selezione, per inserimento e a bolle
  \item Calcolare il numero di confronti e di scambi degli algoritmi di ordinamento basati su confronti e scambi
  \item Comprendere i criteri di scelta di un algoritmo rispetto ad altri
  \item Astrarre il modello di semplici problemi di natura quantitativa e descrivere algoritmicamente il procedimento di soluzione
  \item Simulare l'esecuzione di un programma
\end{enumerate}

Il concetto di algoritmo viene formalizzato usando i concetti di problema, modello, codifica, istanza, soluzione ed esecutore.
%Le attività di apprendimento interattivo fanno ricorso alla manipolazione di carte da gioco, attività estrapolate da \cite{csu} e giochi interattivi presenti nella sezione "Interactives" di \cite{csfg}.
%Vengono condivise delle risorse di studio con la descrizione degli algoritmi in pseudo-linguaggio, diagramma delle attività e due linguaggi di programmazione.
%L'insegnante esegue passo passo gli algoritmi alla lavagna illustrando le variazioni di stato.
%Lo studente è in grado di simulare l'esecuzione di un algoritmo, conosce alcuni problemi di ricerca e ordinamento, calcola la complessità computazionale di un semplice algoritmo basandosi sul numero di volte che una certa operazione viene eseguita, valuta l'algoritmo più efficiente per risolvere un problema noto in base alle caratteristiche dell'istanza.

\subsubsection[Sistemi Operativi]{Interagire col sistema operativo tramite la shell}
\label{sec:sistemi-operativi}

\noindent\textbf{Area tematica}: Sistemi operativi (SO)

%\noindent\textbf{Abilità}:
    
\begin{enumerate}
  \item Creare, rinominare, spostare e cancellare un file
  \item Organizzare i file nelle directory
  \item Invocare un programma
  \item Elencare file e directory
  \item Creare una pipe anonima tra due o più utility
  \item Usare comandi per la data, l'estrazione della testa e della coda di un file di testo, l'ordinamento delle linee, l'estrazione di campi
\end{enumerate}

Le Indicazioni nazionali richiedono la conoscenza delle funzionalità dei sistemi operativi più comuni
e si è scelto di orientare lo studio al sistema operativo GNU/Linux in quanto esso è il sistema operativo
più eseguito\footnote{Si veda \url{https://en.wikipedia.org/wiki/Usage_share_of_operating_systems}}
e meglio documentato, oltre a soddisfare il requisito di essere software libero.
% Nel primo bienno si è limitata l'analisi delle funzionalità al
% concetto di \textit{file system}, processo e di \textit{shell}. Le interfacce grafiche per la gestione dei file sono ben note agli studenti e quindi si è scelto di lavorare sul linguaggio \texttt{bash} e sui filtri più comuni.
%Si è optato per un laboratorio avente PC operanti con la distribuzione FUSS\footnote{Si veda il \url{https://fuss.bz.it/}}.
%Gli studenti sanno condurre operazioni complesse impossibili da effettuare con le interfacce grafiche.

\subsubsection[Web]{Documenti elettronici per il web}
\label{sec:web}

\noindent\textbf{Area tematica}: Elaborazione digitale dei documenti (DE)

%\noindent\textbf{Abilità}:
    
\begin{enumerate}
  \item Riconoscere un linguaggio basato sui marcatori
  \item Usare i tag HTML per creare un documento valido
  \item Usare i tag semantici per strutturare il contenuto
  \item Saper realizzare un semplice sito web statico curando solo il contenuto
  \item Collegare un foglio di stile ad una pagina web
  \item Modificare lo stile del testo
  \item Modificare l'aspetto delle scatole
  \item Saper realizzare un semplice sito web statico curando anche la presentazione
\end{enumerate}

I linguaggi specifici di dominio del web per il contenuto e la presentazione sono introdotti
nel primo biennio ma vengono esplorati anche nel secondo biennio.

\subsection{Secondo biennio}
\label{sec:secondo-biennio}

\subsubsection[Fogli elettronici]{Fogli elettronici}
\label{sec:fogli-elettronici}

\noindent\textbf{Area tematica}: Elaborazione digitale dei documenti (DE)

%\noindent\textbf{Abilità}:
    
\begin{enumerate}
  \item Usare le operazioni di filtraggio, riduzione e mappa nel foglio di calcolo
  \item Usare il foglio di calcolo per modellizzare e risolvere le problematiche d'interesse per il corso di studi
  \item Usare il risolutore per problemi di scelta
  \item Impostare problemi a variabili intere di ammissibilità e di ottimizzazione vincolata con il risolutore
  \item Descrivere l'ordine delle operazioni di aggiornamento in un foglio di calcolo
\end{enumerate}

Si propongono le funzionalità del foglio di calcolo che corrispondono a costrutti tipici della

\subsubsection[Paradigmi]{Paradigmi di programmazione}
\label{sec:paradigmi}

\noindent\textbf{Area tematica}: Algoritmi e linguaggi di programmazione (AL)

%\noindent\textbf{Abilità}:
    
\begin{enumerate}
  \item Saper realizzare algoritmi usando le funzioni e  i costrutti di sequenza, selezione e iterazione (paradigma strutturato)
  \item Riconoscere la ricorsione come caratteristica sintattica
  \item Riconoscere se il processo generato da una procedura sintatticamente ricorsiva è lineare o ricorsivo
  \item Progettare semplici basi di fatti e regole con paradigma logico
  \item Saper realizzare semplici algoritmi mediante funzioni sintatticamente ricorsive
  \item Saper attraversare semplici strutture dati definite ricorsivamente
\end{enumerate}

L'autore non è riuscito a decodificare le richieste ministeriali
di \textit{implementazione di un linguaggio di programmazione,
metodologie di programmazione, sintassi di un linguaggio
orientato agli oggetti}, pertanto si è scelto di presentare il paradigma
della programmazione strutturata (con funzioni definite nell'ambito 
di una funzione), % in JavaScript,
quello logico % con esempi in Prolog,
e quello funzionale. % in JavaScript, realizzando
%con funzioni pure di ordine superiore.
Alla programmazione orientata agli oggetti, richiesta dalle Indicazioni, sono dedicati altri moduli.




\subsubsection[Web app]{Applicazioni web}
\label{sec:web-app}

\noindent\textbf{Area tematica}: Algoritmi e linguaggi di programmazione (AL)

%\noindent\textbf{Abilità}:
    
\begin{enumerate}
  \item Saper scrivere semplici script interpretabili dal web browser
  \item Saper usare le API di accesso al modello di documento di una pagina HTML
  \item Strutturare i dati per permettere la costruzione programmatica di una pagina HTML
  \item Realizzare una semplice web app interattiva
\end{enumerate}

Questo argomento introdurre alcuni concetti della programmazione orientata agli oggetti
quali quello di istanza di classe, interfaccia, metodi e proprietà.

\subsubsection[Grafi]{Problemi su grafi e combinatorici}
\label{sec:grafi}

\noindent\textbf{Area tematica}: Algoritmi e linguaggi di programmazione (AL)

%\noindent\textbf{Abilità}:
    
\begin{enumerate}
  \item Saper rappresentare un grafo
  \item Realizzare gli algoritmi di visita di un albero binario: preordine, postordine, simmetrico% e a livelli
  \item Visitare un grafo in profondità e in ampiezza
  \item Simulare l'algoritmo di Dijkstra
  \item Rappresentare sul foglio di calcolo il problema dello zaino
\end{enumerate}

\subsection{Quinto anno}
\label{sec:quinto-anno}

\subsubsection[Reti]{Reti di calcolatori}
\label{sec:reti}

\noindent\textbf{Area tematica}: Reti di computer (RC)

%\noindent\textbf{Abilità}:
    
\begin{enumerate}
  \item Descrivere i componenti hardware e software che costituiscono una rete dati
  \item Descrivere le applicazioni di rete quali quelle del www e della posta elettronica
  \item Descrivere i livelli di rete del modello ISO/OSI
  \item Riconoscere vantaggi e svantaggi di alcune topologie di rete
\end{enumerate}

\subsubsection[Internet]{Internet e servizi}
\label{sec:internet}

\noindent\textbf{Area tematica}: Struttura di Internet e servizi (IS)

%\noindent\textbf{Abilità}:
    
\begin{enumerate}
  \item Descrivere i servizi di rete con particolare riferimento al www
  \item Descrivere gli strati del modello TCP/IP
  \item Comprendere i principali rischi di sicurezza nelle comunicazioni digitali
  \item Valutare l'uso di tecniche per raggiungere determinati livelli di riservatezza, integrità e disponibilità dei dati
\end{enumerate}

\subsubsection[Funzioni di ordine superiore]{Funzioni di ordine superiore}
\label{sec:funzioni-di ordine superiore}

\noindent\textbf{Area tematica}: Computazione, calcolo numerico e simulazione (CS)

%\noindent\textbf{Abilità}:
    
\begin{enumerate}
  \item Definire funzioni che hanno come argomento una funzione di variabile reale e un valore reale e che restituiscono un valore reale
  \item Definire funzioni che hanno come argomento una funzione di variabile reale e che restituiscono una funzione di variabile reale
  \item Descrivere gli algoritmi in termini di applicazione e composizione di funzioni
  \item Manipolare gli array usando le tecniche: filter, map, reduce.
\end{enumerate}

\subsubsection[Radici]{Radici di funzioni non lineari}
\label{sec:radici}

\noindent\textbf{Area tematica}: Computazione, calcolo numerico e simulazione (CS)

%\noindent\textbf{Abilità}:
    
\begin{enumerate}
  \item Saper applicare il metodo di Erone di Alessandria per il calcolo delle radici quadrate
  \item Riconoscere l'esistenza di una radice di una funzione dato un intervallo
  \item Saper effettuare la ricerca numerica di una radice tramite il metodo di bisezione
  \item Realizzare in un linguaggio di programmazione e al foglio di calcolo il metodo di bisezione
  \item Saper effettuare la ricerca numerica di una radice tramite il metodo delle tangenti (di Newton)
  \item Realizzare in un linguaggio di programmazione e al foglio di calcolo il metodo delle tangenti (di Newton)
  \item Saper riconoscere l'equivalenza di due metodi iterativi
\end{enumerate}

\subsubsection[Ottimizzazione]{Ottimizzazione 1-dimensionale}
\label{sec:ottimizzazione}

\noindent\textbf{Area tematica}: Computazione, calcolo numerico e simulazione (CS)

%\noindent\textbf{Abilità}:
    
\begin{enumerate}
  \item Riconoscere l'esistenza di un massimo e di un minimo
  \item Metodo di Newton per l'ottimizzazione
  \item Approssimare il punto di minimo locale usando il metodo di Newton per l'ottimizzazione
  \item Realizzare un un linguaggio di programmazione e al foglio di calcolo il metodo di Newton per l'ottimizzazione
  \item Realizzare un un linguaggio di programmazione e al foglio di calcolo il metodo della sezione aurea
\end{enumerate}

%%% FINE CURRICOLO

\section{Sperimentazione}
\label{sec:sperimentazione}

L'azione didattica ispirata da tale curricolo è stata condotta tra gli anni scolastici
2018/19 e 2022/23 coinvolgendo due classi, di cui una di una sperimentazione quadriennale.
Entrambe le classi hanno avuto come insegnante, dal primo all'Esame di Stato,
l'autore di questo contributo.

Le misure adottate per fronteggiare l'emergenza sanitaria per COVID-19 nel periodo in cui
si è adottato il curricolo hanno condizionato negativamente sulla sperimentazione e sulla
rilevazione dei risultati.
È stato impossibile condurre un'analisi sul raggiungimento degli obiettivi di competenza per classi
parallele al termine del secondo anno, periodo in cui gli studenti non erano nelle aule scolastiche,
e non si è proceduto al rilevamento nel quarto e quinto anno per via della forte eterogeneità con le
classi parallele che si è avuta anche per l'assenza di didattica.

L'esperienza si è conclusa per l'esigenza di uniformare l'offerta didattica tra classi parallele,
per la difficoltà di reperire validi supporti didattici e per la scarsità del tempo che l'autore
ha trovato per la produzione o la traduzione del materiale didattico.

\section{Conclusioni}
\label{sec:conclusioni}

In conclusione si considera il curricolo qui proposto come
una proposta coerente con i traguardi di competenze fissate dal  
Ministero dell'Istruzione e del Merito che include alcuni elementi
di dettagli della programmazione didattica
orientati alla disciplina scientifica piuttosto che all'uso delle applicazioni.
Tale curricolo può essere una bozza per la progettazione e la programmazione
della didattica dell'informatica % che sia formativa e orientante
alla quale gli insegnati liceali possono attingere per la costruzione del percorso
del proprio istituto.

% Non si possono fornire indicazioni sull'efficacia della proposta a causa della mancanza di uno studio comparativo.
Il curricolo presenta alcune difficoltà di realizzazione in particolare per
la mancanza di materiali di studio, auspicabilmente nella forma di Open Educational Resource (OER),
in lingua italiana e di qualità elevata.

%L'esperienza è però difficile da riprodurre e richiede la produzione di tanto materiale
%didattico da fornire agli studenti.

Questa difficoltà è superabile se gli insegnanti partecipano a comunità professionali
che arricchiscano di OER i progetti esistenti, come quello del Politecnico di Torino~\cite{fare}.
Oltre agli OER servono libri di testo e piattaforme didattiche.
Testi e piattaforme ci sono sono già ma occorre un impegno della comunità nella traduzione
professionale dei migliori progetti didattici già attuati all'estero.

\label{sect:bib}
%\addcontentsline{toc}{section}{\refname}
%\bibliographystyle{plain}
%\bibliographystyle{apacite}
%\bibliography{curricolo_sa}
\printbibliography

\end{document}

\end{document}

% TRADIZIONE SICP
% https://www.composingprograms.com/