% -*- mode: latex; coding: utf8; TeX-master: curricolo_sa.tex -*-
% !TeX root = curricolo_sa.tex
% !TeX encoding = UTF-8 Unicode
% !Tex TS-program = LuaLaTeX
\documentclass[a4paper]{easychair}
\usepackage[italian]{babel}
\usepackage[autostyle,italian=guillemets,]{csquotes}
\usepackage{doc}
\usepackage[utf8]{inputenc}

%% Front Matter
%%
% Regular title as in the article class.
%
\title{Un curricolo per separare l'informatica dall'applimatica\thanks{neologismo %
trovato in \href{https://aladdin.unimi.it/materiali/talk/2012_mirabilandia.pdf}%
{alcuni lucidi del prof. Mattia Monga}}\\ in un %
Liceo scientifico opzione scienze applicate}

\author{
Gionata Massi\institute{
  IIS Savoia Benincasa, Ancona (AN)
 }
}

\institute{
  IIS Savoia Benincasa, Ancona (AN)\\
  \email{gionata.massi@savoiabenincasa.it}
}

\authorrunning{Massi}

\titlerunning{Un curricolo per le Scienze Applicate}

\begin{document}

\maketitle

\begin{abstract}
Questo contributo presenta un percorso pensato per l'insegnamento dell'informatica
nel Liceo scientifico opzione scienze applicate, andando a concepire la materia come
una disciplina scientifico/ingegneristica\footnote{Harold Abelson e 
Gerald Jay Sussman, gli autori di \href{https://web.mit.edu/6.001/6.037/sicp.pdf}{SICP}
potrebbero non essere d'accordo sulla natura dell'informatica} che coniuga creatività e metodo.

Il curricolo si concentra sui traguardi di competenza fissati dalle Indicazioni
nazionali~\cite{indicazioniNazionali}
ma presenta una selezione di argomenti e contenuti
che sconcertano studenti, genitori e colleghi.

Le aspettative sembrano concentrarsi su due grandi stereotipi,
da una parte quello dell'applimatica$\phantom{}^\star$,
ossia del mero uso del computer e delle tecnologie del momento,
dall'altra quella dell'apprendimento di un linguaggio di programmazione, in genere uno e uno solo,
frequentemente il C++\footnote{Bjarne Stroustrup è contento ma anche lui sembra
consapevole del fatto che il linguaggio non tenga in minimo conto le esigenze didattiche.}.
I libri di testo seguono questi stereotipi o aggiungono quello del \textit{coding}.

La sperimentazione ha riguardato due classi e si è conclusa per l'esigenza
di uniformare l'offerta didattica tra classi parallele con insegnanti diversi che,
in mancanza di validi supporti didattici trovano estremamente oneroso sviluppare
ex novo l'attività didattica.

In conclusione si considera la proposta della didattica dell'informatica come
formativa e orientante ma troppo complessa in assenza di 
Open Educational Resource (OER) in lingua italiana e di qualità.
\end{abstract}

\setcounter{tocdepth}{2}
{\small
\tableofcontents}

\section{Introduzione}
\label{sect:introduction}

Gli insegnanti di informatica italiani, a differenza dei colleghi di molte
altre nazioni, non hanno sillabi, programmi nazionali e sistemi di valutazione
sviluppati per l'insegnamento della loro materia ma devono
programmare l'attività didattica mediando fra vincoli normativi e
disparate esigenze.

Negli Stati Uniti ci sono curricola e attività didattiche progettate in
dettaglio, come~\cite{fisler2021evolving} che propone scopi affini
alle indicazioni nazionali~\cite{indicazioniNazionali}.
Le esperienze e i curricula sono spesso raccolti in catalogati come~\cite{cs4all}.
Per i curriculum come lo Advanced Placement Computer Science~\cite{ap-csp}
sono offerti vari corsi gratuiti anche da piattaforme come~\cite{codeorg}
e~\cite{khan-academy}.
La comunità scientifica è attiva nelle proposte didattiche e sviluppa linguaggi
e sistemi ad hoc\footnote{%
Seymour Papert e Mitchel Resnick sono così famosi tra gli insegnati italiani che è inutile citarli.} %
(\cite{abelson1996structure} e \cite{abelson2022structure}, \cite{friedman1995little}).

In Nuova Zelanda hanno sviluppato la \textit{Computer Science Field Guide}~\cite{csfg},
una guida interattiva che così interessante da essere tradotta in varie lingue ma non in italiano.

In Gran Bretagna c'è molta attenzione all'insegnamento dell'informatica
e alla formazione professionale dei docenti\footnote{Un progetto di riferimento è~\cite{nc4ce} ma tante comunità e fondazioni
di produttori di hardware Open Source che propongono curricula completi.
}~\cite{fowler2021england}.

In Italia ogni insegnante, attenendosi alle Indicazioni nazionali e al PTOF
della scuola in cui insegna, considerate le necessità del territorio,
tenuto conto delle aspettative di studenti, dei genitori e dei colleghi e
di eventuali libri testo, 
scrive la sua programmazione didattica.

Si propone una di queste programmazioni costruita rilassando
i vincoli sulle aspettative e sui testi, prevendendo attività
alla lavagna e, se al PC, usando esclusivamente
Software Libero e formati aperti.

%Nel par.~\ref{sec:b1} si propongono alcune abilit\`a
%pensate per il primo biennio, nel par.~\ref{sec:b2} quelle del secondo
%biennio e quinto anno, mentre nel par.~\ref{sec:conclusioni}
%si tirano le somme.

\section{Primo biennio}
\label{sec:b1}

Si propongono alcune attività del primo biennio in sintonia con le indicazioni nazionali.

\subsection[Scrittura tecnico-scientifica]{Scrittura tecnico-scientifica al calcolatore} 

Per soddisfare il requisito per cui `lo studente conosce gli elementi costitutivi
di un documento elettronico e i principali strumenti di produzione' e la
pressante richiesta dei colleghi sull'uso di un sistema di videoscrittura\footnote{Per
l'insegnate tipo il programma di videoscrittura è solo quella versione del noto produttore
di sistemi operativi di cui egli ha ottenuto una qualche licenza}, si è proposta un'unità
didattica di introduzione alla comunicazione tecnico/scientifica per il conseguimento
delle seguenti abilità:

\begin{enumerate}
\item
  Identificare gli attori e di una comunicazione
\item
  Distinguere tra dati, informazioni e conoscenza
\item
  Aggiungere l'autore e il titolo nel frontespizio di una relazione
\item
  Suddividere un testo nei suoi elementi strutturali ed aggiungere i
  titoletti
\item
  Scrivere tabelle, immagini e altri contenuti flottanti con didascalie
\item
  Scrivere con elenchi ordinati, non ordinati e descrittivi
\item
  Enfatizzare il testo selezionando un carattere tipografico opportuno
\end{enumerate}

Si è scelto AsciiDoctor\cite{asciidoctor} con \LaTeX{} e \texttt{asciidoctor-diagram} come caso di studio.
L'autore ha costruito un'applicazione web basata per
la sperimentazione e la verifica automatica del livello di apprendimento.
La didattica del linguaggio è stata condotta con esempi e generalizzazioni
della sintassi tramite diagrammi rail-road.

Come risultato qualche studente ha usato il sistema per produrre
relazioni delle esperienze di fisica scritte e composte tipograficamente
in modo impeccabile.

\subsection[Dati e codifiche]{Dati e codifiche} 

Per soddisfare il requisito sulla `\textellipsis una introduzione alla codifica binaria
\textellipsis i codici ASCII e Unicode', si è focalizzato su:

\begin{enumerate}
  \item
    Convertire un numero da base due a base dieci
  \item
    Convertire un numero da base dieci a base due
  \item
    Codificare un testo usando codifiche binarie dei caratteri
  \item
    Decodificare un testo codificato in binario
  \item
    Comprendere le esigenze che hanno condotto allo sviluppo dello
    standard UNICODE
  \item
    Codificare un'immagine monocromatica mediante run-length encoding
  \item
    Decodificare un'immagine monocromatica codificata mediante run-length
    encoding
\end{enumerate}  

Lo strumento didattico di riferimento è stato il testo
`Computer Science Unplugged`~\cite{csu}.

L'apprendimento delle codifiche è stato reso più attraente rispetto alle modalità
didattiche classiche.

\subsection[Problemi e algoritmi]{Problemi, modelli, soluzioni e algoritmi}

Per introdurre gli studenti alla programmazione vengono prima presentati
e formalizzati i problemi.
Si sono scelte le seguenti abilità:

\begin{enumerate}
  \item
    Saper formalizzare un problema di ricerca
  \item
    Simulare l'esecuzione dell'algoritmo di ricerca lineare
  \item
    Simulare l'esecuzione dell'algoritmo di ricerca binaria
  \item
    Saper formalizzare il concetto di ordinamento di una sequenza
  \item
    Simulare l'algoritmo di ordinamento per selezione, per inserimento e a bolle
  \item
    Calcolare il numero di confronti e di scambi degli algoritmi di
    ordinamento basati su confronti e scambi
  \item
    Comprendere i criteri di scelta di un algoritmo rispetto ad altri
  \item
    Astrarre il modello di semplici problemi di natura quantitativa e
    descrivere algoritmicamente il procedimento di soluzione
  \item
    Simulare l'esecuzione di un programma
\end{enumerate}
  
Anche in questo caso si è fatto ricorso a `Computer Science Unplugged'.

\subsection[Sistemi Operativi]{Interagire col sistema operativo tramite la shell}

Per introdurre il sistema operativo si è scelto di saper interagire con il suo
livello più esterno sviluppando le seguenti abilità:

\begin{enumerate}
  \def\labelenumi{\arabic{enumi}.}
  \item
    Creare, rinominare, spostare e cancellare un file
  \item
    Organizzare i file nelle directory
  \item
    Invocare un programma
  \item
    Elencare file e directory
  \item
    Creare una pipe anonima tra due o più utility
  \item
    Usare comandi per la data, l'estrazione della testa e della coda
    di un file di testo, l'ordinamento delle linee, l'estrazione di campi
\end{enumerate}

Si è fatto ricorso ad una connessione `ssh` ad un server GNU/Linux.

\section{Secondo biennio e quinto anno}

\subsection[Web]{Documenti elettronici per il web}

Nel secondo biennio si è cercato di sviluppare, gradualmente, l'astrazione
e di introdurre vari paradigmi di programmazioni, tranne quello detto
\textit{Banana Gorilla Jungle} perché con due ore a settimana bisogna
fare delle scelte.
 
\subsection[Fogli elettronici]{Fogli elettronici}

I fogli elettronici sono richiesti dai colleghi e si possono
introdurre i concetti di reattività ma anche di funzione di
ordine superiore in modo intuitivo. Si sono stabiliti i seguenti
traguardi:

\begin{enumerate}
  \item
    Usare le operazioni di filtraggio, riduzione e mappa nel foglio di
    calcolo
  \item
    Usare il foglio di calcolo per modellizzare e risolvere le
    problematiche d'interesse per il corso di studi
  \item
    Usare il risolutore per problemi di scelta
  \item
    Impostare problemi a variabili intere con il risolutore
\end{enumerate}

\subsection[Paradigmi]{Paradigmi di programmazione}

Nell'incapacità dell'autore di decodificare le richieste
di `implementazione di un linguaggio di programmazione,
metodologie di programmazione, sintassi di un linguaggio
orientato agli oggetti', si è scelto di presentare il paradigma
della programmazione strutturata (con funzioni definite nell'ambito 
di una funzione) in JavaScript, quello logico con esempi in Prolog,
e quello funzionale in JavaScript, realizzando funzioni pure espresse
senza l'uso dei costrutti \texttt{while}, \texttt{do-while} e texttt{for}.

\begin{enumerate}
  \item\
    Saper realizzare algoritmi usando le funzioni e 
    i costrutti di sequenza, selezione e iterazione (paradigma strutturato)
  \item
    Progettare semplici basi di fatti e regole con paradigma logico
  \item
    Saper realizzare semplici algoritmi mediante funzioni sintatticamente ricorsive
\end{enumerate}

\subsection[Grafi]{Problemi su grafi e combinatorici}

\begin{enumerate}
  \item Saper rappresentare un grafo
  \item
    Realizzare gli algoritmi di visita di un albero binario: preordine,
    postordine, simmetrico % e a livelli
  \item
    Visitare un grafo in profondità
  \item
    Visitare un grafo in ampiezza
  \item
    Simulare l'algoritmo di Dijkstra
  \item Rappresentare sul foglio di calcolo il problema dello zaino
\end{enumerate}

\subsection[Funzioni di ordine superiore]{Funzioni di ordine superiore}

\begin{enumerate}
  \item
    Definire funzioni che hanno come argomento una funzione di variabile
    reale e un valore reale e che restituiscono un valore reale
  \item
    Definire funzioni che hanno come argomento una funzione di variabile
    reale e che restituiscono una funzione di variabile reale
  \item
    Descrivere gli algoritmi in termini di applicazione e composizione di funzioni
  \item
    Manipolare gli array usando le tecniche: filter, map, reduce.
\end{enumerate}

\subsection[Radici]{Radici di funzioni}

\begin{enumerate}
  \item Saper applicare il metodo di Erone di Alessandria per il calcolo delle radici quadrate
  \item
    Riconoscere l'esistenza di una radice di una funzione dato un intervallo
  \item
    Saper effettuare la ricerca numerica di una radice tramite il metodo
    di bisezione
  \item
    Realizzare in un linguaggio di programmazione e al foglio di calcolo
    il metodo di bisezione
    \item
    Saper effettuare la ricerca numerica di una radice tramite il metodo
    delle tangenti (di Newton)
  \item
    Realizzare in un linguaggio di programmazione e al foglio di calcolo
    il metodo delle tangenti (di Newton)
  \item
    Saper riconoscere l'equivalenza di due metodi iterativi
\end{enumerate}

\subsection[Ottimizzazione]{Ottimizazione nel caso di una variabile reale}

\begin{enumerate}
  \item
  Riconoscere l'esistenza di un massimo e di un minimo
  \item
    Metodo di Newton per l'ottimizzazione
  \item
    Approssimare il punto di minimo locale usando il metodo di Newton
  per l'ottimizzazione
  \item
    Realizzare un un linguaggio di programmazione e al foglio di calcolo
  il metodo di Newton per l'ottimizzazione
  \item
    Realizzare un un linguaggio di programmazione e al foglio di calcolo
  il metodo della sezione aurea
\end{enumerate}
  
\section{Conclusioni}

Il percorso proposto mira al conseguimento degli obiettivi proposti dal Ministero
dell'Istruzione e del Merito e, a giudizio dell'autore, forma degli studenti
maggiormente consapevoli e capaci di orientarsi meglio nelle scelte universitarie.

Le tematiche proposte si sono dimostrate stimolanti sia per gli studenti che per
l'insegnante.

L'esperienza è però difficile da riprodurre e richiede la produzione di tanto materiale
didattico da fornire agli studenti. 

Per poter osare con una didattica di questo tipo occorre arricchire di OER i progetti
esistenti, come quello del Politecnico di Torino~\cite{fare}, di tradurre in modo
professionale i migliori progetti didattici già attuati all'estero, di creare
comunità di insegnanti di informatica.

\label{sect:bib}
%\addcontentsline{toc}{section}{\refname}
\bibliographystyle{plain}
%\bibliographystyle{alpha}
%\bibliographystyle{unsrt}
%\bibliographystyle{abbrv}
\bibliography{curricolo_sa}

\end{document}

\end{document}

