
\subsection{Primo biennio}
\label{sec:primo-biennio}

\subsubsection[Scrittura tecnico-scientifica]{Scrittura tecnico-scientifica al calcolatore}
\label{sec:scrittura-tecnico-scientifica}

\noindent\textbf{Area tematica}: Elaborazione digitale dei documenti (DE)

%\noindent\textbf{Abilità}:
    
\begin{enumerate}
  \item Identificare gli attori di una comunicazione
  \item Distinguere tra dati, informazioni e conoscenza
  \item Aggiungere l'autore e il titolo nel frontespizio di una relazione
  \item Suddividere un testo nei suoi elementi strutturali ed aggiungere i titoletti
  \item Scrivere tabelle, immagini e altri contenuti flottanti con didascalie
  \item Scrivere con elenchi ordinati, non ordinati e descrittivi
  \item Enfatizzare il testo selezionando un carattere tipografico opportuno
\end{enumerate}

L'argomento è introdotto per soddisfare sia il vincolo ministeriale sulla produzione di documenti elettronici.
La videoscrittura si colloca di buon diritto nell'applimatica per cui si è scelto l'argomento della comunicazione tecnico/scientifica
e un linguaggio di dominio specifico dotato di grammatica libera dal contesto.
Lo strumento di elaborazione del testo diventa così un linguaggio avente una grammatica espressa tramite il suo diagramma sintattico
e l'attività di videoscrittura diventa un'approccio ai linguaggi formali e alla programmazione.

\subsubsection[Dati e codifiche]{Dati e codifiche}
\label{sec:dati-e codifiche}

\noindent\textbf{Area tematica}: Architettura dei computer (AC)

%\noindent\textbf{Abilità}:
    
\begin{enumerate}
  \item Convertire un numero da base due a base dieci
  \item Convertire un numero da base dieci a base due
  \item Codificare un testo usando codifiche binarie dei caratteri
  \item Decodificare un testo codificato in binario
  \item Comprendere le esigenze che hanno condotto allo sviluppo dello standard UNICODE
  \item Codificare un'immagine monocromatica mediante run-length encoding
  \item Decodificare un'immagine monocromatica codificata mediante run-length encoding
\end{enumerate}

L'argomento è proposto per soddisfare la richiesta di introdurre lo studente alla codifica binaria
senza limitarsi ai codici codici ASCII e Unicode citati nelle Indicazioni nazionali.
Verrà ampiamente ripreso al quinto anno quando, nell'analisi degli algoritmi numerici,
sono trattate le codifiche in virgola mobile e gli errori di troncamento.

%La didattica si avvale delle attività proposte nel testo \textit{Computer Science Unplugged}~\cite{csu}.

%Lo studente non è introdotto alla sola rappresentazione del testo ma anche a quella dei numeri naturali e delle immagini raster. Al concetto di cifra binaria come elemento di codifica viene affiancato in modo intuitivo il concetto di \textit{bit} come unità di misura dell'informazione e la capacità di mettere in relazione la quantità di bit usati per rappresentare la stessa informazione usando codici diversi.

\subsubsection[Problemi e algoritmi]{Problemi, modelli, soluzioni e algoritmi}
\label{sec:problemi-e algoritmi}

\noindent\textbf{Area tematica}: Algoritmi e linguaggi di programmazione (AL)

%\noindent\textbf{Abilità}:
    
\begin{enumerate}
  \item Saper formalizzare un problema di ricerca
  \item Simulare l'esecuzione dell'algoritmo di ricerca lineare
  \item Simulare l'esecuzione dell'algoritmo di ricerca binaria
  \item Saper formalizzare il concetto di ordinamento di una sequenza
  \item Simulare l'algoritmo di ordinamento per selezione, per inserimento e a bolle
  \item Calcolare il numero di confronti e di scambi degli algoritmi di ordinamento basati su confronti e scambi
  \item Comprendere i criteri di scelta di un algoritmo rispetto ad altri
  \item Astrarre il modello di semplici problemi di natura quantitativa e descrivere algoritmicamente il procedimento di soluzione
  \item Simulare l'esecuzione di un programma
\end{enumerate}

Il concetto di algoritmo viene formalizzato usando i concetti di problema, modello, codifica, istanza, soluzione ed esecutore.
%Le attività di apprendimento interattivo fanno ricorso alla manipolazione di carte da gioco, attività estrapolate da \cite{csu} e giochi interattivi presenti nella sezione "Interactives" di \cite{csfg}.
%Vengono condivise delle risorse di studio con la descrizione degli algoritmi in pseudo-linguaggio, diagramma delle attività e due linguaggi di programmazione.
%L'insegnante esegue passo passo gli algoritmi alla lavagna illustrando le variazioni di stato.
%Lo studente è in grado di simulare l'esecuzione di un algoritmo, conosce alcuni problemi di ricerca e ordinamento, calcola la complessità computazionale di un semplice algoritmo basandosi sul numero di volte che una certa operazione viene eseguita, valuta l'algoritmo più efficiente per risolvere un problema noto in base alle caratteristiche dell'istanza.

\subsubsection[Sistemi Operativi]{Interagire col sistema operativo tramite la shell}
\label{sec:sistemi-operativi}

\noindent\textbf{Area tematica}: Sistemi operativi (SO)

%\noindent\textbf{Abilità}:
    
\begin{enumerate}
  \item Creare, rinominare, spostare e cancellare un file
  \item Organizzare i file nelle directory
  \item Invocare un programma
  \item Elencare file e directory
  \item Creare una pipe anonima tra due o più utility
  \item Usare comandi per la data, l'estrazione della testa e della coda di un file di testo, l'ordinamento delle linee, l'estrazione di campi
\end{enumerate}

Le Indicazioni nazionali richiedono la conoscenza delle funzionalità dei sistemi operativi più comuni
e si è scelto di orientare lo studio al sistema operativo GNU/Linux in quanto esso è il sistema operativo
più eseguito\footnote{Si veda \url{https://en.wikipedia.org/wiki/Usage_share_of_operating_systems}}
e meglio documentato, oltre a soddisfare il requisito di essere software libero.
% Nel primo bienno si è limitata l'analisi delle funzionalità al
% concetto di \textit{file system}, processo e di \textit{shell}. Le interfacce grafiche per la gestione dei file sono ben note agli studenti e quindi si è scelto di lavorare sul linguaggio \texttt{bash} e sui filtri più comuni.
%Si è optato per un laboratorio avente PC operanti con la distribuzione FUSS\footnote{Si veda il \url{https://fuss.bz.it/}}.
%Gli studenti sanno condurre operazioni complesse impossibili da effettuare con le interfacce grafiche.

\subsubsection[Web]{Documenti elettronici per il web}
\label{sec:web}

\noindent\textbf{Area tematica}: Elaborazione digitale dei documenti (DE)

%\noindent\textbf{Abilità}:
    
\begin{enumerate}
  \item Riconoscere un linguaggio basato sui marcatori
  \item Usare i tag HTML per creare un documento valido
  \item Usare i tag semantici per strutturare il contenuto
  \item Saper realizzare un semplice sito web statico curando solo il contenuto
  \item Collegare un foglio di stile ad una pagina web
  \item Modificare lo stile del testo
  \item Modificare l'aspetto delle scatole
  \item Saper realizzare un semplice sito web statico curando anche la presentazione
\end{enumerate}

I linguaggi specifici di dominio del web per il contenuto e la presentazione sono introdotti
nel primo biennio ma vengono esplorati anche nel secondo biennio.

\subsection{Secondo biennio}
\label{sec:secondo-biennio}

\subsubsection[Fogli elettronici]{Fogli elettronici}
\label{sec:fogli-elettronici}

\noindent\textbf{Area tematica}: Elaborazione digitale dei documenti (DE)

%\noindent\textbf{Abilità}:
    
\begin{enumerate}
  \item Usare le operazioni di filtraggio, riduzione e mappa nel foglio di calcolo
  \item Usare il foglio di calcolo per modellizzare e risolvere le problematiche d'interesse per il corso di studi
  \item Usare il risolutore per problemi di scelta
  \item Impostare problemi a variabili intere di ammissibilità e di ottimizzazione vincolata con il risolutore
  \item Descrivere l'ordine delle operazioni di aggiornamento in un foglio di calcolo
\end{enumerate}

Si propongono le funzionalità del foglio di calcolo che corrispondono a costrutti tipici della

\subsubsection[Paradigmi]{Paradigmi di programmazione}
\label{sec:paradigmi}

\noindent\textbf{Area tematica}: Algoritmi e linguaggi di programmazione (AL)

%\noindent\textbf{Abilità}:
    
\begin{enumerate}
  \item Saper realizzare algoritmi usando le funzioni e  i costrutti di sequenza, selezione e iterazione (paradigma strutturato)
  \item Riconoscere la ricorsione come caratteristica sintattica
  \item Riconoscere se il processo generato da una procedura sintatticamente ricorsiva è lineare o ricorsivo
  \item Progettare semplici basi di fatti e regole con paradigma logico
  \item Saper realizzare semplici algoritmi mediante funzioni sintatticamente ricorsive
  \item Saper attraversare semplici strutture dati definite ricorsivamente
\end{enumerate}

L'autore non è riuscito a decodificare le richieste ministeriali
di \textit{implementazione di un linguaggio di programmazione,
metodologie di programmazione, sintassi di un linguaggio
orientato agli oggetti}, pertanto si è scelto di presentare il paradigma
della programmazione strutturata (con funzioni definite nell'ambito 
di una funzione), % in JavaScript,
quello logico % con esempi in Prolog,
e quello funzionale. % in JavaScript, realizzando
%con funzioni pure di ordine superiore.
Alla programmazione orientata agli oggetti, richiesta dalle Indicazioni, sono dedicati altri moduli.




\subsubsection[Web app]{Applicazioni web}
\label{sec:web-app}

\noindent\textbf{Area tematica}: Algoritmi e linguaggi di programmazione (AL)

%\noindent\textbf{Abilità}:
    
\begin{enumerate}
  \item Saper scrivere semplici script interpretabili dal web browser
  \item Saper usare le API di accesso al modello di documento di una pagina HTML
  \item Strutturare i dati per permettere la costruzione programmatica di una pagina HTML
  \item Realizzare una semplice web app interattiva
\end{enumerate}

Questo argomento introdurre alcuni concetti della programmazione orientata agli oggetti
quali quello di istanza di classe, interfaccia, metodi e proprietà.

\subsubsection[Grafi]{Problemi su grafi e combinatorici}
\label{sec:grafi}

\noindent\textbf{Area tematica}: Algoritmi e linguaggi di programmazione (AL)

%\noindent\textbf{Abilità}:
    
\begin{enumerate}
  \item Saper rappresentare un grafo
  \item Realizzare gli algoritmi di visita di un albero binario: preordine, postordine, simmetrico% e a livelli
  \item Visitare un grafo in profondità e in ampiezza
  \item Simulare l'algoritmo di Dijkstra
  \item Rappresentare sul foglio di calcolo il problema dello zaino
\end{enumerate}

\subsection{Quinto anno}
\label{sec:quinto-anno}

\subsubsection[Reti]{Reti di calcolatori}
\label{sec:reti}

\noindent\textbf{Area tematica}: Reti di computer (RC)

%\noindent\textbf{Abilità}:
    
\begin{enumerate}
  \item Descrivere i componenti hardware e software che costituiscono una rete dati
  \item Descrivere le applicazioni di rete quali quelle del www e della posta elettronica
  \item Descrivere i livelli di rete del modello ISO/OSI
  \item Riconoscere vantaggi e svantaggi di alcune topologie di rete
\end{enumerate}

\subsubsection[Internet]{Internet e servizi}
\label{sec:internet}

\noindent\textbf{Area tematica}: Struttura di Internet e servizi (IS)

%\noindent\textbf{Abilità}:
    
\begin{enumerate}
  \item Descrivere i servizi di rete con particolare riferimento al www
  \item Descrivere gli strati del modello TCP/IP
  \item Comprendere i principali rischi di sicurezza nelle comunicazioni digitali
  \item Valutare l'uso di tecniche per raggiungere determinati livelli di riservatezza, integrità e disponibilità dei dati
\end{enumerate}

\subsubsection[Funzioni di ordine superiore]{Funzioni di ordine superiore}
\label{sec:funzioni-di ordine superiore}

\noindent\textbf{Area tematica}: Computazione, calcolo numerico e simulazione (CS)

%\noindent\textbf{Abilità}:
    
\begin{enumerate}
  \item Definire funzioni che hanno come argomento una funzione di variabile reale e un valore reale e che restituiscono un valore reale
  \item Definire funzioni che hanno come argomento una funzione di variabile reale e che restituiscono una funzione di variabile reale
  \item Descrivere gli algoritmi in termini di applicazione e composizione di funzioni
  \item Manipolare gli array usando le tecniche: filter, map, reduce.
\end{enumerate}

\subsubsection[Radici]{Radici di funzioni non lineari}
\label{sec:radici}

\noindent\textbf{Area tematica}: Computazione, calcolo numerico e simulazione (CS)

%\noindent\textbf{Abilità}:
    
\begin{enumerate}
  \item Saper applicare il metodo di Erone di Alessandria per il calcolo delle radici quadrate
  \item Riconoscere l'esistenza di una radice di una funzione dato un intervallo
  \item Saper effettuare la ricerca numerica di una radice tramite il metodo di bisezione
  \item Realizzare in un linguaggio di programmazione e al foglio di calcolo il metodo di bisezione
  \item Saper effettuare la ricerca numerica di una radice tramite il metodo delle tangenti (di Newton)
  \item Realizzare in un linguaggio di programmazione e al foglio di calcolo il metodo delle tangenti (di Newton)
  \item Saper riconoscere l'equivalenza di due metodi iterativi
\end{enumerate}

\subsubsection[Ottimizzazione]{Ottimizzazione 1-dimensionale}
\label{sec:ottimizzazione}

\noindent\textbf{Area tematica}: Computazione, calcolo numerico e simulazione (CS)

%\noindent\textbf{Abilità}:
    
\begin{enumerate}
  \item Riconoscere l'esistenza di un massimo e di un minimo
  \item Metodo di Newton per l'ottimizzazione
  \item Approssimare il punto di minimo locale usando il metodo di Newton per l'ottimizzazione
  \item Realizzare un un linguaggio di programmazione e al foglio di calcolo il metodo di Newton per l'ottimizzazione
  \item Realizzare un un linguaggio di programmazione e al foglio di calcolo il metodo della sezione aurea
\end{enumerate}
